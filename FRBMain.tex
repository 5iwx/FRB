%!TeX program=xelatex
%! BIB program = bibtex
\documentclass[zihao=-4]{ctexart}
\usepackage[normalem]{ulem}
\useunder{\uline}{\ul}{}
%********************导言区宏包引入********************
\usepackage{xeCJK}
\usepackage{amssymb}
\usepackage{amsmath}
\usepackage{listings} %代码
\usepackage{graphicx}
\usepackage{multicol} %回车换段
\usepackage{xcolor}
\usepackage{geometry} %页面设置
\usepackage{fontspec}
\usepackage{setspace}
\usepackage{times}
\usepackage{fancyhdr} %页眉页脚
\pagestyle{fancy}
\usepackage{float} %表格位置
\usepackage{titlesec} %设置
\usepackage{titletoc}
\usepackage{ctex}
\usepackage{gbt7714}    %控制参考文献格式为国标
\usepackage{multirow}
\usepackage{booktabs}   %表格相关
\usepackage{setspace}   %设置行距
\usepackage{caption} %caption
\usepackage{subcaption} %子图的caption
\usepackage{changepage} %左右缩进


\graphicspath{ {include_picture/} }
\let\algorithm\relax
\let\endalgorithm\relax
\usepackage[ruled,vlined]{algorithm2e}%[ruled,vlined]{
\usepackage{algpseudocode}
\renewcommand{\algorithmicrequire}{\textbf{Input:}} 
\renewcommand{\algorithmicensure}{\textbf{Output:}}
%\renewcommand\thepage{\zihao{-5} ~\arabic{page}~}%页码字号

%定义两个arg
\DeclareMathOperator*{\argmax}{arg\,max}
\DeclareMathOperator*{\argmin}{arg\,min}
\DeclareCaptionLabelSeparator{mysep}{\space\space}  %自定义caption格式
\captionsetup[figure]{labelfont=bf, labelsep=mysep, textfont=bf}   %图片caption格式
\captionsetup[table]{labelfont=bf, labelsep=mysep, textfont={bf}}   %表格caption格式
\bibliographystyle{gbt7714-numerical} %修改了title斜体内容

%********************导言区宏包引入********************
%********************第三方字体引入********************
%\setCJKmainfont[Path=fonts/,BoldFont=simhei.ttf,ItalicFont=simkai.ttf,SlantedFont=simfang.ttf]{simsun.ttc}
%中文字体涵盖黑体、宋体、楷体、仿宋
\setmainfont[Path=fonts/, 
BoldFont = times-new-roman-bold.ttf,
ItalicFont = times-new-roman-italic.ttf,
BoldItalicFont = times-new-roman-bold-italic.ttf
]{times-new-roman.ttf}
\setmonofont[Path=fonts/]{Courier New.ttf}
\setCJKfamilyfont{hwzs}[Path=fonts/]{STKzhongsong.ttf}%使用STZhogsong华文中宋字体
\newcommand{\zhongsong}{\CJKfamily{hwzs}}
\setCJKfamilyfont{hwxw}[Path=fonts/]{STKxinwei.ttf} % XSP 2023/3/3:
\newcommand{\xinwei}{\CJKfamily{hwxw}}              %  使用STZxinwei华文新魏字体.

%********************第三方字体引入********************


%********************中文字号设置********************
%\newcommand{\chuhao}{\fontsize{42pt}{\baselineskip}\selectfont}
\newcommand{\chuhao}{\fontsize{42pt}{0}}
\newcommand{\xiaochu}{\fontsize{36pt}{0}}
\newcommand{\yihao}{\fontsize{28pt}{0}}
\newcommand{\erhao}{\fontsize{21pt}{0}}
\newcommand{\xiaoer}{\fontsize{18pt}{0}}
\newcommand{\sanhao}{\fontsize{16pt}{0}}
\newcommand{\sihao}{\fontsize{14pt}{0}}
\newcommand{\xiaosi}{\fontsize{12pt}{0}}
\newcommand{\wuhao}{\fontsize{10.5pt}{0}}
\newcommand{\xiaowu}{\fontsize{9pt}{0}}
\newcommand{\liuhao}{\fontsize{8pt}{0}}
\newcommand{\qihao}{\fontsize{5.25pt}{0}}
%********************中文字号设置********************


%********************页边距设置********************
\geometry{left=3cm,right=2cm,top=2.5cm,bottom=2.5cm}
\geometry{a4paper} % xsp 2023/3/7: 调整纸张大小为A4
%********************页边距设置********************

%********************段间距设置********************
\newcommand{\setParDis}{\setlength {\parskip} {0pt} }
%请在每部分使用这个
%********************段间距设置********************

%********************

\begin{document}
%********************页眉页脚设置********************
\lhead{}%设置左页眉为空
\rhead{}%设置左页眉为空
\setlength{\headwidth}{\textwidth}% 2023/3/3 XSP: 页眉长度适应文本
%********************页眉页脚设置********************


%********************标题格式设置********************

%\setcounter{secnumdepth}{0}%该命令取消了章标题前数字label

\CTEXsetup[name={,、},number={\chinese{section}}]{section}
\CTEXsetup[name={(,)},number={\chinese{subsection}}]{subsection}
\CTEXsetup[name={,.},number={\arabic{subsubsection}}]{subsubsection}% 不加会导致目录格式错误
% 设置subsubsection等格式
% \titleformat{\section}[block]{\sanhao\bfseries\centering}{\chinese{section}、}{0pt}{}[]
% \titleformat{\subsection}[block]{\sihao\bfseries}{(\chinese{subsection})}{0pt}{}[]
% \titleformat{\subsubsection}[block]{\xiaosi\bfseries}{\arabic{subsubsection}、}{0pt}{}[]
\titleformat{\section}[block]{\sanhao\heiti\centering}{\chinese{section}、}{0pt}{}[]    % XSP 2023/3/3:
\titleformat{\subsection}[block]{\sihao\heiti}{(\chinese{subsection})}{0pt}{}[]       %   将正文标题字体由加粗
\titleformat{\subsubsection}[block]{\xiaosi\heiti}{\arabic{subsubsection}.}{0pt}{}[]   % 修改为黑体。
\titlespacing{\section}{0pt}{25pt}{12pt}
\titlespacing{\subsection}{0pt}{7pt}{7pt}
\titlespacing{\subsubsection}{0pt}{5pt}{4pt}

\titlecontents{section}[1.6em]{\addvspace{2pt}\filright}
{\contentspush{\thecontentslabel\hspace{0.8em}}}
{}{\titlerule*[8pt]{.}\contentspage}

\titlecontents{subsection}[3.2em]{\addvspace{2pt}\filright}
{\contentspush{\thecontentslabel\hspace{0.8em}}}
{}{\titlerule*[8pt]{.}\contentspage}

\titlecontents{subsubsection}[6.4em]{\addvspace{2pt}\filright}
{\contentspush{\thecontentslabel\hspace{0.8em}}}
{}{\titlerule*[8pt]{.}\contentspage}
%********************标题格式设置********************

%\setcounter{section}{-3}  %标题计数器
%\stepcounter{section}

%*******************行间距段前段后*******************
\linespread{1.8}
%行间距为实际行间距乘以1.2,如此处实际为1.5倍行距
\setlength{\parskip}{0.5\baselineskip}
%*******************行间距段前段后*******************



%********************封面部分********************
%
%     论文题目:应准确、鲜明、简洁,能概括整个论文中最主要和最重要的内容。
% 题目不超过20个中文字,若语意未尽,可用副标题补充说明。副标题应处于从属
% 地位,一般可在题目的下一行用破折号“——”引出。论文题目应避免使用不常用缩
% 略词、首字母缩写字、字符、代号和公式等。
%
\leftline{\includegraphics[scale=1]{include_picture/xiaohui.png}} % XSP 2023/3/3: 取消校徽段首缩进
%格式控制部分
% \par \  
% \par \
% \par \
\vspace{32pt}
\begin{center}
\includegraphics[height=2.25cm, width=12.78cm, scale=1]{include_picture/xiaoming.png}
\end{center}
%格式控制部分
\vspace{12pt}

\begin{spacing}{3}
    % \erhao
    \begin{center}
        \zhongsong\erhao{第三十三届“冯如杯”竞赛主赛道项目论文模板} %黑体这样调用,其余字体同理
        
        % \zhongsong{“冯如杯”竞赛主赛道项目是什么}
    \end{center}
    \rightline{\xinwei\sanhao{——基于 Latex 的论文模板}} % XSP 2023/3/3: 副标题二号华文新魏居右
\end{spacing}
%格式控制部分
% \par \ 
% \par \
\par \ 
\par \
\par \ 
\par \
% \begin{center}
%     \sihao
%     \textbf{学院:计算机学院}
%     \par \ 
%     \textbf{本模板原作者:Someday}
% \end{center}

%格式控制部分
\par \ 
\begin{center}
\sanhao
\centerline{\heiti{}}%封面年月去掉
\end{center}

\pagenumbering{gobble} %封面无页码
%\thispagestyle{empty}


\renewcommand{\headrulewidth}{0pt}%没有页眉装饰线
\clearpage
\pagenumbering{roman} %摘要目录页小写罗马

\xiaosi
\section*{摘要}
\begin{spacing}{1.5}
  \setParDis %设置段间距为 0
  本Latex模板是北京航空航天大学大学第三十三届“冯如杯”竞赛主赛道论文模板, 由北京航空航天大学校团委
基于GitHub用户\textbf{\textit{Somedaywilldo}}与\textbf{\textit{cpfy}}的成果迭代
开发而来。在此由衷感谢所有开发者对本模板的贡献与对“冯如杯”竞赛的大力支持。

摘要内容包括:“摘要”字样,摘要正文,关键词。在摘要的最下方另起一行,用显著的字符注明文本的关键词。

摘要是论文内容的简短陈述,应体现论文工作的核心思想。摘要一般约500字。摘要内容应涉及本项科研工作的目的和意义、研究思想和方法、研究成果和结论。

关键词是为用户查找文献,从文中选取出来用来揭示全文主题内容的一组词语或术语,应尽量采用词表中的规范词(参照相应的技术术语标准)。关键词一般为3到8个,按词条的外延层次排列。关键词之间用逗号分开,最后一个关键词后不打标点符号。

\end{spacing}
    
\textbf{关键词:}关键词1,关键词2,关键词3,关键词4,关键词5

\newpage
\section*{\textbf{Abstract}} % XSP 2023/3/8: Abstract 加粗
\begin{spacing}{1.5}
\begin{adjustwidth}{0.42cm}{0.42cm}
  \setParDis %设置段间距为 0

\qquad This Latex template for the 33rd Fengru Cup Competition of 
Beihang University, is developed by Communist Youth League Committee of BUAA 
iteratively based on the contribution of GitHub 
users \textbf{\textit{Somedaywilldo}} and \textbf{\textit{cpfy}}. 
Here, we would like to thank all the developers for their 
contributions to this template and for their support of the Fengru Cup Competition.

The abstract includes: the word "Abstract", the body of the abstract, and the keywords. On a separate line at the bottom of the abstract, indicate the key words of the text in prominent characters.

The abstract is a short statement of the content of the paper and should reflect the core ideas of the paper work. The abstract is usually about 500 words. The abstract should cover the purpose and significance of this scientific work, research ideas and methods, research results and conclusions.

Keywords are a set of words or terms selected from the text to reveal the subject content of the whole text for the user to find the literature, and the standardized words in the word list (refer to the corresponding technical terminology standards) should be used as much as possible. The keywords are usually 3 to 8, arranged according to the level of extensibility of the words. The keywords are separated by commas, and no punctuation marks are used after the last keyword.

\textbf{Keywords: Keywords 1, Keywords 2, Keywords 3, Keywords 5, Keywords 6}
\end{adjustwidth}
\end{spacing}



%********************摘要部分********************


%********************目录部分********************
\clearpage
\tableofcontents
\clearpage
%********************目录部分********************



\renewcommand{\headrulewidth}{0.4pt} %恢复页眉装饰线

%********************正文页眉部分********************
%\lhead{} 
\chead{\xiaowu 北京航空航天大学第三十三届“冯如杯”竞赛主赛道参赛作品} %设置居中页眉
%********************正文页眉部分********************

\pagenumbering{arabic} %正文页码从1开始,用阿拉伯数字
\setcounter{page}{1} 

\section{简介}
\setParDis %设置段间距为 0
\begin{spacing}{1.5}
  第三十三届“冯如杯”主赛道论文一律由在计算机上输入、排版、定稿后转成PDF格式,
在集中申报时通过网络上传。\textbf{论文封面及全文中不能出现作者姓名、学院、专业、指导老
师的相关信息}。包括5个部分, 顺序依次为: \par 
  % 段落间可加入\par进行换行,代替两行回车的写法
  \begin{itemize}
    \item 封面(中文)
    \item 中文摘要、关键词(中文、英文)
    \item 主体部分
    \item 结论
    \item 参考文献
  \end{itemize}

\section{论文的书写规范}

论文正文部分需分章节撰写,每章应另起一行。
章节标题要突出重点,简明扼要、层次清晰。字数一般在15字以内,不得使用标点符号。标题中尽量不采用英文缩写词,对必须采用者,应使用本行业的通用缩写词。  层次以少为宜,根据实际需要选择。三级标题的层次按章(如“一、”)、节(如 “(一)”)、条(如“1.”)的格式编写,各章题序的阿拉伯数字用 Times New Roman 体。  

\subsection{字体和字号}
论文题目:二号,华文中宋体加粗,居中。

副标题:三号,华文新魏,居右(可省略)。

章标题:三号,黑体,居中。

节标题:四号,黑体,居左。

条标题:小四号,黑体,居左。

正文:小四号,中文字体为宋体,西文字体为Times New Roman体,首行缩进,两端对齐。

页码:五号Times New Roman 体,数字和字母\par

\subsection{页边距及行距}
学术论文的上边距:25mm;下边距:25mm;左边距:30mm;右边距 20mm。
章、节、条三级标题为单倍行距,段前、段后各设为0.5行(即前后各空0.5行)。
正文为 1.5 倍行距,段前、段后无空行(即空0行)。

\subsection{页眉}
页眉内容为北京航空航天大学第三十三届“冯如杯”竞赛主赛道参赛作品,内容居中。
页眉用小五号宋体字,页眉标注从论文主体部分开始(引言或第一章)。
请注意论文封面无页眉。

\subsection{页码}
论文页码从“主体部分(引言、正文、结论)”开始,直至“参考文献”结束,用五号阿
拉伯数字连续编码,页码位于页脚居中。\textbf{封面、题名页不编页码。}

摘要、目录、图标清单、主要符号表用五号小罗马数字连续编码,页码位于页脚居中。

\subsection{图、表及其附注}
图和表应安排在正文中第1次提及该图、表的文字的下方,当图或表不能安排在该页时,
应安排在该页的下一页。

\subsubsection{图}
图题应明确简短,\textbf{用五号宋体加粗},数字和字母为\textbf{五号Times New Roman体
加粗},图的编号与图题之间应空半角2格。图的编号与图题应置于图下方的居中位置。图内文字
为\textbf{5号宋体},数字和字母为\textbf{5号Times New Roman体}。曲线图的纵横坐标必
须标注“量、标准规定符号、单位”,此三者只有在不必要注明(如无量刚等)的情况下方可省略。
坐标上标注的量的符号和缩略词必须与正文中一致。

\subsubsection{表}
表的标号应采用从1开始的阿拉伯数字编号,如:“表 1”、“表 2”、……。表编号应一直连续到附录
之前,并与章、节和图的编号无关。只有一幅表,仍应标为“表 1”。表题应明确简短,用\textbf{五号宋体
加粗},数字和字母为\textbf{五号Times New Roman体加粗},表的编号与表题之间应空半角2格。表的编号与
表头应置于表上方的居中位置。表内文字为\textbf{5号宋体},数字和字母为\textbf{5号Times New Roman体}。  

\subsubsection{附注}
图、表中若有附注时,附注各项的序号一律用“附注+阿拉伯数字+冒号”,如:“附注1:”。

附注写在图、表的下方,一般采用5号宋体。

\subsubsection{参考文献}
凡有直接引用他人成果(文字、数字、事实以及转述他人的观点)之处,均应加标注说明列于参考文献中,
以避免论文抄袭现象的发生。

标注格式:引用参考文献标注方式应全文统一,标注的格式为[序号],放在引文或转述观点的最后
一个句号之前,所引文献序号用小4号Times New Roman体、以上角标形式置于方括号中,如“……成果”$^{[1]}$。

\section{原理与算法}
\subsection{基于哈达玛积的问题转化}
\subsection{范围证明的理论基础}
\subsubsection{默克尔树}
默克尔树(Merkle Tree or Hash Tree)是一棵用哈希值搭建起来的树,树的所有节点都存
储了哈希值。整棵树包含根节点、中间节点和叶节点。树采取自下而上的生成方式,叶节点经
哈希运算得到哈希值,而其余节点的哈希值均由其子节点的哈希值经哈希计算得到。默克尔树
的具体结构见图\ref{Merkle}。
\begin{figure}[H]
  \centering
  \includegraphics[width=0.8\textwidth]{Merkle_tree.bmp} 
  \caption{默克尔树结构图}
  \label{Merkle}
\end{figure}
基于哈希函数的防碰撞特性(Collision resistance)、隐藏性(Hiding)和谜题
友好性(Puzzle friendly),对默克尔树的任意局部修改,都会对根节点和路径
上的中间节点产生影响。默克尔树的这个特性提供了一种很好的检测数据是否被篡
改的方法。在本文中,我们使用具有抗碰撞特性和不可逆特性的哈希函数来构造默
克尔树,进而利用构造的树来完成对向量的承诺,并通过次线性尺度的证明来开放
树的多处索引。对向量$\;v\;$的承诺包含以下三种算法,即承诺操作(Commit)、
开放操作(Open)、验证操作(Verify):
\begin{itemize}
  \item $\text{root}_v\leftarrow \text{MT.Commit}(v)$
  \item $(\{v_i\}_{i\in\mathcal{I}},\pi^v_{\mathcal{I}})\leftarrow\text{MT.Open}(\mathcal{I},v)$
  \item $\{1,0\}\leftarrow\text{MT.Verify}(\text{root}_v,\mathcal{I},\{v_i\}_{i\in\mathcal{I}},\pi^v_{\mathcal{I}})$
\end{itemize}

\subsubsection{里得·所罗门编码}
里得·所罗门编码(Reed-Solomon Code, RS Code)是一种编码方式,用其编码的码字是域上
某个特定单变量多项式的一组函数值,表示成向量的形式。因此,在本文中我们使用RS\;Code来编码
向量。\par
用抽象代数的模型来定义,选择一个$\;q\;$阶的有限域$\mathbb{F}$,作为编码的字母表。
再选择$\mathbb{F}$的一个陪集$\;L\;$,所选择的特定单变量多项式成为编码多项式(encoding polynomial),且度
小于$\rho\cdot|L|$,其中$\rho\in(0,1)$称为编码率,用这样的多项式编码出的向量表示为
$\text{RS}[L,\rho]\in\mathbb{F}^{|L|}$。\par
具体而言,编码的过程如下:设插值集$H=\{\xi_1,\cdots,\xi_{|H|}\}$,估值集$L=\{\eta_1,\cdots,\eta_{|L|}\}$,
且$|L|>|H|$,被编码的向量设为$v\in\mathbb{F}^{|H|}$。首先,找到预设的度的编码多项式$\hat{p}$,使得
$\hat{p}|_H=\{\hat{p}(\xi_1),\cdots,\hat{p}(\xi_{|H|})\}=v$。然后计算$\hat{p}$在$L$上的估值(Evaluation),得到
码字$\hat{p}|_L$。计算估值和插值的算法使用快速傅里叶变换(Fast Fourier Transform, FFT)和其逆变换(Inverse FFT, IFFT)。

\subsubsection{诚实验证方前提的零知识证明}
零知识证明(Zero-Knowledge Argument of Knowledge, ZKAoK
)是一种验证协议,在其中证明方(Prover)不提供任何有关某个论断的有用信息,
而能使验证方(Verifier)验证该论断为正确的。这项协议技术
在信息安全及密码学等领域应用广泛。
“诚实验证方前提(Honest Verifier)”意为验证方是正确遵循协议进行验证的。\par
用计算复杂度理论(Computational Complexity Theory)的模型定义,零知识证明
是一个用于证明NP(Non-deterministic Polynomial)二元关系$\mathcal{R}$的
算法三元组$(\mathcal{G}, \mathcal{P}, \mathcal{V})$。其中$\mathcal{G}$表示
公共参数生成算法,设其输出为pp;$\mathcal{P}$和$\mathcal{V}$分别表示非确定多
项式时间(Probabilistic Polynomial Time, PPT)的证明算法和验证算法。\par
诚实验证方前提的零知识证明具有以下条件需要满足:
\begin{itemize}
  \item \textbf{\emph{完备性(Completeness)}}:即正确的论断都可以被证明为正确。假设$\lambda$为私有参数,
      对于每个$\mathcal{G}$的输出pp$\leftarrow\mathcal{G}(1^{\lambda})$,每
      个$\mathcal{R}$中的元素$(x,\omega)$,以及字母表上任意字符串$z\in\{0,1\}^*$,有:
      \[\text{Pr}[\langle \mathcal{P}({\omega}),\mathcal{V}(z)(\text{pp},x)=1\rangle]=1-\text{negl}(\lambda)\]
      其中Pr表示概率,negl($\lambda$)表示当$\lambda$足够大时,可以忽略不计的量。
  \item \textbf{\emph{正确性(Soundness)}}:即被证明的论断大都是正确的,只有极小的可能出错。对于每个$\mathcal{G}$的输出pp
      $\leftarrow\mathcal{G}(1^{\lambda})$,
      每个不在$\mathcal{R}$中的元素$(x,\omega)$,以及字母表上任意字符串$z\in\{0,1\}^*$,有:
      \[\text{Pr}[\langle \mathcal{P}^*({\omega}),\mathcal{V}(z)(\text{pp},x)=1\rangle]\le\text{negl}(\lambda)\]
      其中$\mathcal{P}^*$表示任意的PPT证明方。
  \item \textbf{\emph{零知识性(Zero-knowledge)}}:即$\mathcal{P}$和$\mathcal{V}$之间的对话可以只依据公开信息被完全模拟。对于
      每个$\mathcal{G}$的输出pp$\leftarrow\mathcal{G}(1^{\lambda})$,任意的诚实的PPT验证方$\mathcal{V}$,
      每个$\mathcal{R}$中的元素$(x,\omega)$和任意字母表上的字符串$z\in\{0,1\}^*$,存在一个PPT模拟机$\mathcal{S}$,使得:
      \[\{\langle \mathcal{P}(\omega),\mathcal{V}(z)\rangle(\text{pp},x)\}\overset{c}{\approx}\{\mathcal{S}^{\mathcal{V}}(\text{pp},x,z)\}\]
      其中$\mathcal{S}^{\mathcal{V}}$表示多项式空间下给定$\mathcal{V}$的模拟机,$\overset{c}{\approx}$表示两者在计算上不可区分
      (Computationally indistinguishable)。
  \item \textbf{\emph{知识论证性(Argument of knowledge)}}:即所有论证的证明都不会是不合法的。对于每个$\mathcal{G}$的输出pp$\leftarrow\mathcal{G}(1^{\lambda})$,
      任意的$x,z\in\{0,1\}^*$,对于所有恶意的PPT证明方$\mathcal{P^*}$,存在一个可预期多项式时间的抽取机$\mathcal{E}$,使得:
      \[\text{Pr}[\langle \mathcal{P^*}(\omega),\mathcal{V}(z)\rangle(\text{pp},x)=1\wedge((x,\omega)\not\in\mathcal{R})|_{\omega\leftarrow\mathcal{E}^{\mathcal{P^*}}(\text{pp},x)}]\le\text{negl}(\lambda)\]
      其中$\mathcal{E}^{\mathcal{P}^*}$表示抽取机对$\mathcal{P^*}$的任意性及整个运行过程都具有访问权限。
\end{itemize}

\subsubsection{交互式谕示机证明}
交互式谕示机证明(Interactive Oracle Proof, IOP)是一种证明系统模型,在其中验证方可以通过谕示机概率性地问询证明方所持有的
关于被证明的论断的有效信息,但由于是概率性地问询,所以验证方并不能得到证明方的全部信息。\par
同样,使用计算复杂度理论的模型来定义,IOP是证明$\;k\;$轮NP二元关系的算法三元组$(\mathcal{G},\mathcal{P},\mathcal{V})$,其中
$\mathcal{G}$表示公共参数生成算法,设其输出为pp;$\mathcal{P}$和$\mathcal{V}$分别表示PPT证明算法和验证算法。具体而言,一个$\;k\;$轮的IOP包含$\;k\;$
轮的交互(interaction)。在第$\;i\;$轮($0<i\le k$),验证方向证明方均匀且随机地发送消息$m_i$,且验证方能够通过谕示机得到
以$\;m_i\;$为输入的输出,证明方需返回$\;\pi_i\;$给验证方。在最后一轮,验证方得到了证明方返回的$\;k\;$个位置的信息
$\;\pi=(\pi_1,\cdots,\pi_k)$,并且需决定接受或拒绝证明方的证明(Proof)。\par
交互式谕示机证明具有以下条件需要满足:
\begin{itemize}
  \item \textbf{\emph{完备性(Completeness)}}:对于每个pp$\leftarrow\mathcal{G}(1^{\lambda})$以及$(x,\omega)\in\mathcal{R}$,有:
        \[\text{Pr}[\langle \mathcal{P}(\omega),\mathcal{V}^{\pi}\rangle(\text{pp},x)=1]=1\]
        其中$\mathcal{V}^{\pi}$表示$\mathcal{V}$可以访问谕示$\pi$。
  \item \textbf{\emph{正确性(Soundness)}}:对于每个pp$\leftarrow\mathcal{G}(1^{\lambda})$,每个PPT的$\mathcal{P^*}$以及$(x,\omega)\not\in\mathcal{R}$,有:
        \[\text{Pr}[\langle \mathcal{P}^*(\omega),\mathcal{V}^{\pi}\rangle(\text{pp},x)=1]\le\text{negl}(\lambda)\]
\end{itemize}\par
在本文中,主要涉及两种类型的IOP,分别是\emph{RS-IOP}和\emph{IOP of proximity},前者即使用里德-所罗门码(Reed-Solomon code)的IOP,
后者指对于正确性的条件,允许证明方的秘密和合法证据之间具有微小的差距(proximity)。

\subsubsection{单变量求和校验协议}
单变量求和校验协议(Univariate Sum-check Protocol)主要应用于有限域上的多项式求和问题。首先,假设有两个乘法群
$H, L\subset \mathbb{F}\;(|L|>|H|)$,一个小于$k(k>|H|)$阶的单变量多项式$f(\cdot)$,以及一个被声明(claimed)的和$\mu$。
在已有假设上,单变量求和校验协议的作用就是证明$\sum_{a\in H}f(a)=\mu$。\par
在实际操作中,证明方需要将$f(x)$利用带余除法唯一地转化为$x\cdot\hat{p}(x)+\zeta+\hat{Z_H}(x)\hat{h}(x)$,
其中除式为$\hat{Z_H}(x)$,代表$H$上的“消失”多项式(Vanishing polynomial),满足$\forall a\in H,\;\hat{Z_H}(a)=0$。
接着,基于对$\hat{f}|_L$和$\hat{h}|_L$的谕示机访问,验证方可以验证是否有$\hat{p}|_L\in\text{RS}(L,\frac{|H|-1}{|L|})$
以及$\hat{h}|_L\in\text{RS}(L,\frac{|L|-|H|}{|L|})$,其中:
\begin{equation}\hat{p}(x)=\dfrac{|H|\cdot\hat{f}(x)-\mu-|H|\cdot\hat{Z_H}(x)\hat{h}(x)}{x}\end{equation}
以上采用RS编码的IOP满足正确性和完备性,当我们将它转换为一个标准IOP时,它仍然是在检验
谕示$\hat{f}|_L,\hat{h}|_L,\hat{p}|_L$是否为具有相应
度的界限的RS码。而这个过程可以通过下面的低度检测协议来完成。

\subsubsection{低度检测和FRI}
给定度$k_1,\cdots,k_t$,码字$\hat{v}_1|_L,\cdots,\hat{v}_t|_L$,其中$L$为一个乘法陪集,低度检测协议允许验证方借助对这些码字的谕示机访问来检验以下语句是否成立:
\begin{equation}\forall j\in\{0,\cdots,t\},\;\hat{v}_j|_L\in\text{RS}[L,\frac{k_j}{|L|}]\label{低度检测}\end{equation}
公式(\ref{低度检测})用于检测编码给定码字的编码多项式的度是否低于给定的度。\par
在本文中,我们的低度检测协议选取快速Reed-Solomon交互式谕示机邻近证明(Fast Reed-Solomon Interactive Oracle Proof of Proximity, Fast RS IOPP, FRI)。
给定对证明方消息$\;l\;$处取值的谕示机访问,该FRI 是具有完整性(Completeness)和正确性(Soundness)容错率为
$O(\frac{L}{\mathbb{F}})+\text{negl}(l,k)$的IOPP,其中$l=O(\lambda),k=\max\{k_1,\cdots,k_t\}$。\par
总的来说,用于实现单变量求和校验的FRI协议可以表示为:
\begin{equation}\langle \text{FRI}.\mathcal{P}(\hat{f},\hat{h},\hat{p},\text{FRI}.\mathcal{V}^{\hat{f}|_L,\hat{h}|_L,\hat{p}|_L})\rangle(k,k-|H|,|H|-1)\end{equation}\par

\subsubsection{向量内积论证}
向量内积论证(Inner Product Arguments, IPA),是一
证明手段,即给出两个向量$\vec{a},\vec{b}$的承诺
(commitment),其中$\vec{a},\vec{b}$属于
$\mathbb{F}^n$,$\mathbb{F}$为域,可以证明这两个
被承诺的向量的内积等于某一公开的标量,而不需要揭示
这两个向量的具体取值。\par
在信息安全领域,常见的承诺方式有皮特森哈希值
(Pedersen hash)或者RS编码,在本文中采用后者。\par
向量内积论证可以用于证明单变量多项式在某点处的值。首先将多项式
$\hat{f}=f_0+f_1x+\cdots+f_nx^n$表示为向量$\vec{f}=(f_0,f_1,\cdots,f_n)$
注意到:
\begin{equation}\hat{f}(s)=(\vec{f},(1,s,\cdots,s^n))\label{IPA转化}\end{equation}\par
公式(\ref{IPA转化})表明计算等价于两个向量的内积,因此可转化为向量内积论证。

\subsection{基于交互式谕示机证明的零知识范围证明}
\subsubsection{批处理向量内积论证}
为了实现批处理IPA(Batch Inner Product Argument, B-IPA),我们考虑单变量求和校验协议的
一个性质:无论多个多项式的阶数是否相同,单变量求
和校验协议都支持校验每一个一元多项式的和。此性质
为构造一个校验编码向量间内积关系的IPA提供了一种
可能,其中编码向量来自于不同阶数的编码多项式。\par
特别地,将阶数分别为$k_1,...,k_t$的秘密编码多项式
设为$\hat{v}_1,...,\hat{v}_t$。再将阶数分别为
$k_{t+1},...,k_{2t}$的公开多项式设为
$\hat{r}_1,...,\hat{r}_t$。假定证明方$\mathcal{P}$
想要证明对于任意$j \in \{1,\cdots,t\}$,都满足
$\sum_{a\in H}\hat{v}_j(a)\cdot\hat{r}_j(a)=y_j$。
实现过程中,证明方$\mathcal{P}$首先需要用默克尔
树生成对$(\hat{v}_1|_L,...,\hat{v}_t|_L)$的承
诺,并将其发送给验证方$\mathcal{V}$。接着,验证方选
择随机$t$个元素$\beta_1,...,\beta_t$,设
$\hat{q}=\sum\limits_{j=1}^t\beta_j\hat{v}_j\cdot\hat{r}_j$。
最后证明方$\mathcal{P}$和验证方$\mathcal{V}$
使用单变量求和校验协议来证明以下等式成立:
\begin{equation}\sum_{a\in H}\hat{q}(a)=\sum_{a\in H}\sum\limits_{j=1}^t\beta_j\hat{v}_j(a)\cdot\hat{r}_j(a)=\sum\limits_{j=1}^t\beta_jy_j\label{B-IPA和校验}\end{equation}\par
除此之外,批处理IPA的正确性容错率(Soundness error)仅取决于$\;t\;$个项
中最大的阶数$k_{\max}$,$k_{\max}=\max\{k_i+k_{t+i}\}_{1\le i\le t}$。\par
基于此,我们给出\textbf{批处理内积关系\emph{(Batch inner product relation)}}的定义:设二元关系$\mathcal{R}_{\text{B-IPA}}$为所有$(x,\omega)$的集合,其中:
\[\begin{aligned}&x=(\mathbb{F},H,L,\{k_j\}_{1\le j\le2t},\{\hat{r}_j\}_{1\le j\le t},\{y_j\}_{1\le j\le t})\\&\omega=\{\hat{v}_j\}_{1\le j\le t}\end{aligned}\]\par
且有公式(\ref{B-IPA和校验})成立。\par
接下来验证该批处理IPA的正确性、完备性以及知识论证性:\par
\begin{itemize}
  \item \textbf{批处理IPA的完备性\emph{(B-IPA Completeness)}}:考虑$\hat{q}$的变换,
        设对$j \in\{1,\cdots,t\}$,有$\sum_{a\in H}\beta_jv_j(a)r_j(a)=\beta_jy_j$,那么公式(\ref{B-IPA和校验})成立。
        这符合单变量和校验的二元关系形式。因此,批处理IPA有着与单变量和校验协议相同的完备性。
  \item \textbf{批处理IPA的正确性\emph{(B-IPA Soundness)}}:可以考虑以下两种发生错误的情形:\par
        \underline{情形一}.假设由于随机的线性选择组合,非法的单变量和校验关系恰好成立。
        我们假设$\forall j \in\{1,\cdots,t\},\sum_{a\in H}\hat{v}_j(a)\cdot\hat{r}_j(a)=y_j^{\prime}$
        ,且对于$\{1,\cdots,t\}$的某个子集$Q$,有$\forall q \in Q,y_q\neq y_q^{\prime}$。简便起见,不妨设$t\in Q$。
        验证方随机选择t-1个元素$\beta_1,\cdots,\beta_{t-1}$,则$\sum_{j=1}^t\beta_jy_j=\sum_{j=1}^t\beta_jy_j^{\prime}$
        当且仅当:
        \begin{equation}\beta_t=\dfrac{\beta_1(y_1^{\prime}-y_1)+\cdots+\beta_{t-1}(y_{t-1}^{\prime}-y_{t-1})}{y_t-y_t^{\prime}} \label{错误情形一}\end{equation}
        公式(\ref{错误情形一})发生的可能性仅为$1/|\mathbb{F}|$,而实际选用的有限域大小往往很大,因此概率可忽略不计。\par
        \underline{情形二}.假设变量和校验关系是非法的,即公式(\ref{B-IPA和校验})不成立。
          那么批处理IPA正确性的错误有以下三种可能:\par
        (1)若RS编码的IOP非法,则正确性取决于单变量和校验协议,故具有正确性。\par
        (2)若FRI非法,则正确性错误的上界为$\epsilon_{FRI}=\mathcal{O}(|L|/|\mathrm{F}|)+negl(\ell,k_{max}/|L|)$。\par
        (3)若默克尔树的根不正确或任意验证路径不正确,由于哈希函数的防碰撞性质,正确性错误的上界为$negl(\lambda)$。
  \item \textbf{批处理IPA的知识论证性\emph{(B-IPA Knowledge Argument)}}:批处理IPA是基于随机谕示机模型的一种知识论证。
        对于任意PPT对手$\mathcal{P}^\ast$,总存在一个PPT抽取机$\mathcal{E}$使得:给定$\mathcal{P}^\ast$的随机访问带,
        对每个由$\mathcal{P}^\ast$生成的陈述:
        \begin{equation}
          x=(\mathbb{F},H,L,\{k_j\}_{j\in [2t]},\{\hat{r_j}\}_{j\in [t]},\{y_j\}_{j \in [t]})
        \end{equation}\par
        有以下的概率为$\text{negl}(\lambda)$:
        \begin{equation}
          \text{Pr}\left[\begin{matrix}&\text{root}^*\leftarrow\mathcal{P}^*(1^{\lambda},x),\langle\mathcal{P}^*,\mathcal{V}\rangle(\text{pp,x})=1,\{\hat{v}_j\}_{1\le j\le t}\leftarrow\mathcal{E}(1^{\lambda},x):\\&\text{MT.Commit}(\mathbb{V}|_L)\neq\text{root}^*\vee(x,\{\hat{v}_j\}_{1\le j\le t})\not\in\mathcal{R}_{\text{B-IPA}}\end{matrix}\right]
        \end{equation}
        批处理IPA的知识论证属性来源于默克尔树的可抽取性。给定默克尔树树根和足够多的验证通路,总存在一个高效的方法
        能够抽取默克尔树上所有被承诺的叶节点。一旦这些叶节点被成功提取,就能通过IFFT算法获取满足$|L|>k_{\max}$的
        秘密多项式,进而实现知识论证的属性。
\end{itemize}

\subsubsection{对于$[0,u^{n-1}]$的范围证明}
为了证明上界为$u^m$(即$u$进制展开为$m$位)的秘密值$V$在
范围$[0,2^n-1]$中($n<m$),只需要满足以下等式:
\begin{equation}
  \begin{aligned}
    &v\odot(v-1^m)\odot\cdots\odot(v-u^m)=0^m,\\
    &v\odot(1^{m-n}||0^n)=0^m.
  \end{aligned}
\end{equation}\par
其中$v=(v_0,v_1,\ldots,v_{m-1}),V=\sum\limits_{j=0}^mv_ju^j$。进一步地,相当于证明:
\begin{equation}
  \begin{aligned}
    &\langle v\odot(v-1^m)\odot\cdots\odot(v-u^m),r\rangle=0,\\
    &\langle v,r_{[:m-n]}||0^n\rangle=0.
    \label{任意基底范围证明}
  \end{aligned}
\end{equation}\par
经理论计算,对于验证方选取的任意$r\in\mathbb{F}$,公式(\ref{任意基底范围证明})非法成立,即出现正确性错误的概率为$1/\mathbb{F}$。\par
对于公式(\ref{任意基底范围证明}),可以利用批处理IPA来证明,即输入设为:
\begin{equation}
  \begin{aligned}
    &x=(\mathbb{F},H,L,(u|H|-(u-1)),|H|,|H|,|H|,(\hat{r},\hat{s}),(0,0)),\\
    &\omega=(\hat{w},\hat{v})
  \end{aligned}
\end{equation}\par
基于此,我们给出\textbf{范围关系\emph{(Range relation)}}的定义:设二元关系$\mathcal{R}_{\text{RP}}$为
所有$(x,\omega)$的集合,其中:
\[
  x=(\mathbb{F},H,L,m,n,[0,u^n-1]),\;\omega=V.
\]\par
且有秘密值$V$满足$V\in[0,u^n-1]$成立。

\subsubsection{批处理范围证明}
基于前述的批处理IPA,我们可以构建一个证明多个秘密值在各自对应范围内的证明。
假设我们要证明秘密值$V_1,...,V_t$分别处于对应范围$[0,u_1^{n_1}-1],...,[0,u_t^{n_t}-1]$,
那么对于任意$j \in \{1,\ldots,t\}$,都有如下公式:
\begin{equation}\begin{aligned}&v_j\odot(v_j-1^m)\odot\cdots\odot(v_j-u_j^m)=0^m,\\&v_j\odot(1^{m-n_j}||0^{n_j})=0^m\end{aligned}\end{equation}\par
其中$m\ge\max\{n_1,...,n_t\}$。进一步转换上述公式,可以推出:对于任意$j \in \{1,\cdots,t\}$,都有如下公式:
\begin{equation}\begin{aligned}
&\langle v_j\odot(v_j-1^m)\odot...\odot(v_j-u_j^m),r\rangle=0,\\
&\langle v_j,(r^{m-n_j}||0^{n_j})\rangle=0\label{第2次调用B-IPA}
\end{aligned}\end{equation}\par
对于公式(\ref{第2次调用B-IPA}),可以利用批处理IPA来证明,即输入设为:
\begin{equation}\begin{aligned}
&x=(\mathbb{F},H,L,\{k_j\}_{j \in [4t]},\{\hat{r}_j\}_{j\in [2t]},\{y_j\}_{j \in [2t]}),\\
&w=\hat{w}_1,...,\hat{w}_t,\hat{v}_1,...,\hat{v}_t,
\end{aligned}\end{equation}\par
其中一些变量满足如下关系:
\begin{equation}\begin{aligned}
&\{k_j\}_{j\in [t]}=\underbrace{u_1|H|-(u_1-1),...,t_t|H|-(u_t-1)}_t,\\
&\{k_j\}_{j\in [t+1,4t]}=\underbrace{|H|,...,|H|}_{3t},\\
&\{\hat{r}_j\}_{j\in [2t]}=\underbrace{\hat{r},...,\hat{r}}_t,\underbrace{\hat{s}_1,...,\hat{s}_t}_t,\\
&\{y_j\}_{j\in [2t]}=\underbrace{0,...,0}_{2t}.
\end{aligned}\end{equation}\par
其中对任意$j\in \{1,\cdots,t\},\hat{s}_j$和$\hat{v}_j$是$r^{m-n_j}||0^{n_j},v_j$的编码多项式,
以及$\hat{w}_j=\hat{v}_j(\hat{v}_j-1)\cdot\cdot\cdot(\hat{v}_j-u_j)$。\par
\subsubsection{对于任意范围的范围证明}
考虑到实际整数范围证明往往是任意整数,我们需要进一步拓宽前述范围证明的通用性。
为了实现这一点,我们需要利用前述的对范围$[0,u^n-1]$的范围证明。
假设要验证秘密值$V\in[A,B-1]$,其中$A,B$均为任意整数,我们首先将这个问题进行如下转化:
\begin{equation}V-A\in [0,u^n-1]\wedge V-B+u^n\in [0,u^n-1]\end{equation}\par
其中$u^{n-1}<B<u^n$。基于新的公式,我们成功将任意整数范围的证明转化到了
范围$[0,u^n-1]$的证明上,从而我们可以利用前述的对范围$[0,u^n-1]$的范围证明来
实现任意范围的范围证明。\par
在任意范围的证明流程中,验证方不采用基于秘密值$V$的向量$v$的询问,
而是直接让证明方通过IPA证明$V-A=C$和$V-B+u^n=D$。由于此处IPA不止一个,
因此我们可以引入批处理IPA来加快处理过程。实际上,任意范围的范围证明可以
简单看作基于批处理IPA的多个任意基底范围证明的有效融合。
\subsubsection{零知识范围证明}

% 公式示例展示如下:
% \begin{align}
% \underset{{\bf I}_c \sim \boldsymbol\Im_c}{\mathrm{\text P}}
%  \Big( \mathcal C({\bf I}_c) \neq \mathcal C({\bf I}_c + \boldsymbol \rho) \Big) \geq \delta ~~~\text{s.t.}~~||\boldsymbol\rho||_p \leq \xi,
% \label{eq:def1}
% \end{align}

\section{图表模板}
图表示例展示如下:

%插入一张图片
\begin{figure}[H] %H为当前位置,!htb为忽略美学标准,htbp为浮动图形
    \centering %图片居中
    \includegraphics[width=0.8\textwidth]{example-image-2.png} %插入图片,[]中设置图片大小,{}中是图片文件名
    \caption{example\_caption} %最终文档中希望显示的图片标题
    \label{example_label} %用于文内引用的标签
\end{figure}

% 一行三张子图并排示意
\begin{figure}[htbp]
  \begin{subfigure}{0.31\textwidth}
    \includegraphics[width=\linewidth]{example-image-1.png}
    \caption{示意图1} \label{fig:9aaa}
  \end{subfigure}%
  \hspace*{\fill}   % maximize separation between subfigures
  \begin{subfigure}{0.31\textwidth}
    \includegraphics[width=\linewidth]{example-image-1.png}
    \caption{示意图2} \label{fig:9bbb}
  \end{subfigure}
  \hspace*{\fill}   % maximizeseparation between subfigures
  \begin{subfigure}{0.31\textwidth}
    \includegraphics[width=\linewidth]{example-image-1.png}
    \caption{示意图3} \label{fig:9ccc}
    \end{subfigure}
\caption{一行三张子图并排示意}
\label{qmix-train}
\end{figure}


% 2*2四张子图示意
\begin{figure}[htbp]
  \begin{subfigure}{0.48\textwidth}
    \includegraphics[width=\linewidth]{example-image-1.png}
    \caption{示意图1} \label{fig:8a}
  \end{subfigure}%
  \hspace*{\fill}   % maximize separation between subfigures
  \begin{subfigure}{0.48\textwidth}
    \includegraphics[width=\linewidth]{example-image-1.png}
    \caption{示意图2} \label{fig:8b}
  \end{subfigure}\\
  %\hspace*{\fill}   % maximizeseparation between subfigures
  \begin{subfigure}{0.48\textwidth}
    \includegraphics[width=\linewidth]{example-image-1.png}
    \caption{示意图3} \label{fig:8c}
  \end{subfigure}
  \hspace*{\fill}   % maximize separation between subfigures
  \begin{subfigure}{0.48\textwidth}
    \includegraphics[width=\linewidth]{example-image-1.png}
    \caption{示意图4} \label{fig:8d}
  \end{subfigure}
\caption{2*2四张子图示意} \label{fig:8}
\end{figure}

%插入一个表格

\begin{table*}[H]
\centering
\caption{表格使用示例}
\begin{tabular}{|l||c|c|c|c|c|c|}
\hline
{\textbf{表头 1}}     &  表头 2 	    & 表头 3 	  & 表头 4     & 表头 5 		\\ \hline\hline
内容 11 		          & 内容 12				& 内容 13		& 内容 14 	 & 内容 15		\\ \hline
内容 21	              & 内容 22				& 内容 23		& 内容 24	   & 内容 25		\\ \hline


\end{tabular}
\end{table*}


%再插入另一种表格
\begin{table}[H]
\caption{三线表使用示例}
\centering
\begin{tabular}{ccccc}
\hline  
\textbf{方法} & \textbf{表头 1} & \textbf{表头 2} & \textbf{表头 3} & \textbf{表头 4} \\ 
\hline  
方法1 & 数据  & 数据  & 数据  & 数据\\
方法2 & 数据  & 数据  & 数据  & 数据\\
\hline
\end{tabular} 
\end{table} 

\section*{结论}% section*生成无标号章节
\addcontentsline{toc}{section}{结论} % 将无标号章节添加至目录
论文的结论单独作为一章,但不加章号。

注意: 文件大小不超过5M。

\end{spacing}

\newpage

% XSP 2023/3/16: bib支持不全,暂时改为手动
\section*{参考文献} % section*生成无标号章节题目
\addcontentsline{toc}{section}{参考文献} % 将无标号章节添加至目录
% 著作: [序号]作者.书名[标识码].出版地:出版社,出版年.
[1]张志建.严复思想研究[M].桂林:广西师范大学出版社,1989. 

% 译著: [序号]国名或地区(用圆括号)原作者.书名[标识码].译者.出版地:出版社,出版年.
[2](英)霭理士.性心理学[M].潘光旦译.北京:商务印书馆,1997.

% 古典文献 文史古籍类引文后加序号,再加圆括号,内加注书名、篇名

% 论文集: [序号]编者.书名[标识码].出版地:出版社,出版年.
[3]伍蠡甫.西方论文选(下册)[C].上海:上海译文出版社,1979.

% 期刊文章: [序号]作者.篇名[标识码].刊名,年,(期).
[4]叶朗.《红楼梦》的意蕴[J].北京大学学报(哲学社会科学版),1989,(2)

% 报纸文章: [序号]作者.篇名[标识码].报纸名,出版日期(版次)
[5]谢希德.创造学习的新思路[N].人民日报,1998-12-25(10)

% 外文文献: 要求外文文献所表达的信息和中文文献一样多,但文献类型标识码可以不标出。
[6]Mansfeld, R.S. \& Busse. \textit{T.V. The Psychology of creativity and discovery}, Chinago:
NelsonHall, 1981




% \begingroup
% \setstretch{2.0}    %行距2
% \setlength{\bibsep}{0pt}    %段前段后0
% \begin{adjustwidth}{0.42cm}{0.42cm} %左右缩进0.42cm
% \bibliography{references}
% \end{adjustwidth}
% \endgroup

\end{document}
